\section{The preamble of a document}

The screengrab above shows Overleaf storing a \LaTeX\ document as a file called \verb|main.tex|: the \verb|.tex| file extension is, by convention, used when naming files containing your document’s LaTeX code.

The previous example showed how document content was entered after the \\\verb|\begin{document}| command; however, everything in your \verb|.tex| file appearing \emph{before} that point is called the \emph{preamble}, which acts as the document’s “setup” section. Within the preamble you define the document class (type) together with specifics such as languages to be used when writing the document; loading \emph{packages} you would like to use (more on this later), and it is where you’d apply other types of configuration.

A minimal document preamble might look like this:

\begin{tcolorbox}
\begin{verbatim}
    \documentclass[12pt, letterpaper]{article}
    \usepackage{graphicx}
\end{verbatim}
\end{tcolorbox}

where \verb|\documentclass[12pt, letterpaper]{article}| defines the overall class (type) of document. Additional parameters, which must be separated by commas, are included in square brackets (\verb|[...]|) and used to configure this instance of the article class; i.e., settings we wish to use for this particular \verb|article|-class-based document.

In this example, the two parameters do the following:

\begin{itemize}
    \item \verb|12pt| sets the font size
    \item \verb|letterpaper| sets the paper size
\end{itemize}

Of course other font sizes, \verb|9pt|, \verb|11pt|, \verb|12pt|, can be used, but if none is specified, the default size is \verb|10pt|. As for the paper size, other possible values are \verb|a4paper| and \verb|legalpaper|. For further information see the article about \href{https://www.overleaf.com/learn/latex/Page_size_and_margins}{page size and margins}.

The preamble line

\begin{tcolorbox}
\begin{verbatim}
    \usepackage{graphicx}
\end{verbatim}
\end{tcolorbox}

is an example of loading an external package (here, \href{https://ctan.org/pkg/graphicx?lang=en}{graphicx}) to extend \LaTeX’s capabilities, enabling it to import external graphics files. \LaTeX\ packages are discussed in the section \hyperref[sec:16]{Finding and using \LaTeX\ packages}.