\section{Why learn \LaTeX?}

Various arguments can be proposed for, or against, learning to use \LaTeX\ instead of other document-authoring applications; but, ultimately, it is a personal choice based on preferences, affinities, and documentation requirements.

Arguments in favour of \LaTeX\ include:

\begin{itemize}
    \item support for typesetting extremely complex mathematics, tables and technical content for the physical sciences;
    \item facilities for footnotes, cross-referencing and management of bibliographies;
    \item ease of producing complicated, or tedious, document elements such as indexes, glossaries, table of contents, lists of figures;
    \item being highly customizable for bespoke document production due to its intrinsic programmability and extensibility through thousands of \href{https://www.ctan.org/pkg}{free add-on packages}.
\end{itemize}

Overall, \LaTeX\ provides users with a great deal of control over the production of documents which are typeset to extremely high standards. Of course, there are types of documents or publications where \LaTeX\ doesn’t shine, including many “free form” page designs typically found in magazine-type publications.

One important benefit of \LaTeX\ is the separation of document content from document style: once you have written the content of your document, its appearance can be changed with ease. Similarly, you can create a \LaTeX\ file which defines the layout/style of a particular document type and that file can be used as a template to standardise authorship/production of additional documents of that type; for example, this allows scientific publishers to create article templates, in \LaTeX\, which authors use to write papers for submission to journals. Overleaf has a \href{https://www.overleaf.com/gallery}{gallery containing thousands of templates}, covering an enormous range of document types—everything from scientific articles, reports and books to CVs and presentations. Because these templates define the layout and style of the document, authors need only to open them in Overleaf—creating a new project—and commence writing to add their content.