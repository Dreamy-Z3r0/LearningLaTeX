\section{Creating tables}

The following examples show how to create tables in LaTeX, including the addition of lines (rules) and captions.

\subsection{Creating a basic table in \LaTeX}

We start with an example showing how to typeset a basic table:

\begin{tcolorbox}
\begin{verbatim}
    \begin{center}
    \begin{tabular}{c c c}
    cell1 & cell2 & cell3 \\ 
    cell4 & cell5 & cell6 \\  
    cell7 & cell8 & cell9    
    \end{tabular}
    \end{center}
\end{verbatim}
\end{tcolorbox}

This example produces the following output:

\begin{mdframed}
    \begin{center}
    \begin{tabular}{c c c}
    cell1 & cell2 & cell3 \\ 
    cell4 & cell5 & cell6 \\  
    cell7 & cell8 & cell9    
    \end{tabular}
    \end{center}
\end{mdframed}

The \verb|tabular| environment is the default \LaTeX\ method to create tables. You must specify a parameter to this environment, in this case \verb|{c c c}| which advises \LaTeX\ that there will be three columns and the text inside each one must be centred. You can also use \verb|r| to right-align the text and \verb|l| to left-align it. The alignment symbol \verb|&| is used to demarcate individual table cells within a table row. To end a table row use the \emph{new line} command \verb|\\|. Our table is contained within a center environment to make it centred within the text width of the page.

\subsection{Adding borders}

The \verb|tabular| environment supports horizontal and vertical lines (rules) as part of the table:

\begin{itemize}
    \item to add horizontal rules, above and below rows, use the \verb|\hline| command
    \item to add vertical rules, between columns, use the vertical line parameter \verb+|+
\end{itemize}

In this example the argument is \verb+{|c|c|c|}+ which declares three (centred) columns each separated by a vertical line (rule); in addition, we use \verb|\hline| to place a horizontal rule above the first row and below the final row:

\begin{tcolorbox}
\begin{verbatim}
    \begin{center}
    \begin{tabular}{|c|c|c|} 
    \hline
    cell1 & cell2 & cell3 \\ 
    cell4 & cell5 & cell6 \\ 
    cell7 & cell8 & cell9 \\ 
    \hline
    \end{tabular}
    \end{center}
\end{verbatim}
\end{tcolorbox}

\newpage
This example produces the following output:

\begin{mdframed}
    \begin{center}
    \begin{tabular}{|c|c|c|} 
    \hline
    cell1 & cell2 & cell3 \\ 
    cell4 & cell5 & cell6 \\ 
    cell7 & cell8 & cell9 \\ 
    \hline
    \end{tabular}
    \end{center}
\end{mdframed}

Here is a further example:

\begin{tcolorbox}
\begin{verbatim}
    \begin{center}
    \begin{tabular}{||c c c c||} 
    \hline
    Col1 & Col2 & Col2 & Col3 \\ [0.5ex] 
    \hline\hline
    1 & 6 & 87837 & 787 \\ 
    \hline
    2 & 7 & 78 & 5415 \\
    \hline
    3 & 545 & 778 & 7507 \\
    \hline
    4 & 545 & 18744 & 7560 \\
    \hline
    5 & 88 & 788 & 6344 \\ [1ex] 
    \hline
    \end{tabular}
    \end{center}
\end{verbatim}
\end{tcolorbox}

This example produces the following output:

\begin{mdframed}
    \begin{center}
    \begin{tabular}{||c c c c||} 
    \hline
    Col1 & Col2 & Col2 & Col3 \\ [0.5ex] 
    \hline\hline
    1 & 6 & 87837 & 787 \\ 
    \hline
    2 & 7 & 78 & 5415 \\
    \hline
    3 & 545 & 778 & 7507 \\
    \hline
    4 & 545 & 18744 & 7560 \\
    \hline
    5 & 88 & 788 & 6344 \\ [1ex] 
    \hline
    \end{tabular}
    \end{center}
\end{mdframed}

\textbf{Tip!}

\begin{itemize}
    \item Creating tables in \LaTeX\ can be time-consuming so you may want to use the \url{TablesGenerator.com} online tool to export \LaTeX\ code for tabulars.
\end{itemize}

\subsection{Captions, labels and references}

You can caption and reference tables in much the same way as images. The only difference is that instead of the \textbf{figure} environment, you use the \textbf{table} environment.

\begin{tcolorbox}
\begin{verbatim}
    Table \ref{table:data} shows how to add a table caption 
    and reference a table.
    \begin{table}[h!]
    \centering
    \begin{tabular}{||c c c c||} 
    \hline
    Col1 & Col2 & Col2 & Col3 \\ [0.5ex] 
    \hline\hline
    1 & 6 & 87837 & 787 \\ 
    2 & 7 & 78 & 5415 \\
    3 & 545 & 778 & 7507 \\
    4 & 545 & 18744 & 7560 \\
    5 & 88 & 788 & 6344 \\ [1ex] 
    \hline
    \end{tabular}
    \caption{Table to test captions and labels.}
    \label{table:data}
    \end{table}
\end{verbatim}
\end{tcolorbox}

This example produces the following output:

\begin{table}[h!]
\begin{mdframed}
    \paragraph{\-\hspace{20pt}Table \ref{table:data} shows how to add a table caption and reference a table.\\\mbox{}\\}
    
    \centering
    \begin{tabular}{||c c c c||} 
    \hline
    Col1 & Col2 & Col2 & Col3 \\ [0.5ex] 
    \hline\hline
    1 & 6 & 87837 & 787 \\ 
    2 & 7 & 78 & 5415 \\
    3 & 545 & 778 & 7507 \\
    4 & 545 & 18744 & 7560 \\
    5 & 88 & 788 & 6344 \\ [1ex] 
    \hline
    \end{tabular}
    \caption{Table to test captions and labels.}
    \label{table:data}
\end{mdframed}
\end{table}