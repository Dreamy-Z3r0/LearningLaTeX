\section{What is \LaTeX?}

\LaTeX\ (pronounced “\emph{LAY}-tek” or “\emph{LAH}-tek”) is a tool for typesetting professional looking documents. However, LaTeX’s mode of operation is quite different to many other document-production applications you may have used, such as Microsoft Word or LibreOffice Writer: those “\href{https://en.wikipedia.org/wiki/WYSIWYG}{WYSIWYG}” tools provide users with an interactive page into which they type and edit their text and apply various forms of styling. LaTeX works very differently: instead, your document is a plain text file interspersed with LaTeX \emph{commands} used to express the desired (typeset) results. To produce a visible, typeset document, your LaTeX file is processed by a piece of software called a \emph{TeX engine} which uses the commands embedded in your text file to guide and control the typesetting process, converting the LaTeX commands and document text into a professionally typeset PDF file. This means you only need to focus on the \emph{content} of your document and the computer, via LaTeX commands and the TeX engine, will take care of the \emph{visual appearance} (formatting).