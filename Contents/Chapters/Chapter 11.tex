\section{Adding math to \LaTeX}

One of the main advantages of \LaTeX\ is the ease with which mathematical expressions can be written. \LaTeX\ provides two writing modes for typesetting mathematics:

\begin{itemize}
    \item \emph{inline} math mode used for writing formulas that are part of a paragraph
    \item \emph{display} math mode used to write expressions that are not part of a text or paragraph and are typeset on separate lines
\end{itemize}

\subsection{Inline math mode}

Let’s see an example of inline math mode:

\begin{tcolorbox}
\begin{verbatim}
    \documentclass[12pt, letterpaper]{article}
    \begin{document}
    In physics, the mass-energy equivalence is stated 
    by the equation $E=mc^2$, discovered in 1905 by Albert Einstein.
    \end{document}
\end{verbatim}
\end{tcolorbox}

This example produces the following output:

\begin{mdframed}
    \-\hspace{20pt}In physics, the mass-energy equivalence is stated by the equation $E=mc^2$, discovered in 1905 by Albert Einstein.
\end{mdframed}

To typeset inline-mode math you can use one of these delimiter pairs: \verb|\( ... \)|, \verb|$ ... $| or \verb|\begin{math} ... \end{math}|, as demonstrated in the following example:

\begin{tcolorbox}
\begin{verbatim}
    \documentclass[12pt, letterpaper]{article}
    \begin{document}
    \begin{math}
    E=mc^2
    \end{math} is typeset in a paragraph using inline math mode
    ---as is $E=mc^2$, and so too is \(E=mc^2\).
    \end{document}
\end{verbatim}
\end{tcolorbox}

This example produces the following output:

\begin{mdframed}
    \-\hspace{20pt}\begin{math}
    E=mc^2
    \end{math} is typeset in a paragraph using inline math mode---as is $E=mc^2$, and so too is \(E=mc^2\).
\end{mdframed}

\subsection{Display math mode}

Equations typeset in display mode can be numbered or unnumbered, as in the following example:

\begin{tcolorbox}
\begin{verbatim}
    \documentclass[12pt, letterpaper]{article}
    \begin{document}
    The mass-energy equivalence is described by the famous equation
    \[ E=mc^2 \] discovered in 1905 by Albert Einstein. 

    In natural units ($c = 1$), the formula expresses the identity
    \begin{equation}
    E=m
    \end{equation}
    \end{document}
\end{verbatim}
\end{tcolorbox}

This example produces the following output:

\begin{mdframed}
    \-\hspace{20pt}The mass-energy equivalence is described by the famous equation
    \[ E=mc^2 \] discovered in 1905 by Albert Einstein. 

    \-\hspace{20pt}In natural units ($c = 1$), the formula expresses the identity
    \begin{equation}
    E=m
    \end{equation}
\end{mdframed}

To typeset display-mode math you can use one of these delimiter pairs: \verb|\[ ... \]|, \verb|\begin{displaymath} ... \end{displaymath}| or \verb|\begin{equation} ...|\\ \verb|\end{equation}|. Historically, typesetting display-mode math required use of \verb|$$| characters delimiters, as in \verb|$$ ... |\emph{display math here}\verb| ...$$|, but this method is no longer recommended: use LaTeX’s delimiters \verb|\[ ... \]| instead.

\subsection{More complete examples}

The following examples demonstrate a range of mathematical content typeset using LaTeX.

\begin{tcolorbox}
\begin{verbatim}
    \documentclass{article}
    \begin{document}
    Subscripts in math mode are written as $a_b$ and superscripts 
    are written as $a^b$. These can be combined and nested to write
    expressions such as

    \[ T^{i_1 i_2 \dots i_p}_{j_1 j_2 \dots j_q} = 
    T(x^{i_1},\dots,x^{i_p},e_{j_1},\dots,e_{j_q}) \]
 
    We write integrals using $\int$ and fractions using $\frac{a}{b}$. 
    Limits are placed on integrals using superscripts and subscripts:

    \[ \int_0^1 \frac{dx}{e^x} =  \frac{e-1}{e} \]

    Lower case Greek letters are written as $\omega$ $\delta$ etc. 
    while upper case Greek letters are written as $\Omega$ $\Delta$.

    Mathematical operators are prefixed with a backslash as
    $\sin(\beta)$, $\cos(\alpha)$, $\log(x)$ etc.
    \end{document}
\end{verbatim}
\end{tcolorbox}

This example produces the following output:

\begin{mdframed}
    \-\hspace{20pt}Subscripts in math mode are written as $a_b$ and superscripts are written as $a^b$. These can be combined and nested to write expressions such as

    \[ T^{i_1 i_2 \dots i_p}_{j_1 j_2 \dots j_q} = T(x^{i_1},\dots,x^{i_p},e_{j_1},\dots,e_{j_q}) \]
 
    \-\hspace{20pt}We write integrals using $\int$ and fractions using $\frac{a}{b}$. Limits are placed on integrals using superscripts and subscripts:

    \[ \int_0^1 \frac{dx}{e^x} =  \frac{e-1}{e} \]

    \-\hspace{20pt}Lower case Greek letters are written as $\omega$ $\delta$ etc. while upper case Greek letters are written as $\Omega$ $\Delta$.

   \-\hspace{20pt}Mathematical operators are prefixed with a backslash as $\sin(\beta)$, $\cos(\alpha)$, $\log(x)$ etc.
\end{mdframed}

The next example uses the \verb|equation*| environment which is provided by the \verb|amsmath| package, so we need to add the following line to our document preamble:

\begin{tcolorbox}
\begin{verbatim}
    \usepackage{amsmath}% For the equation* environment
\end{verbatim}
\end{tcolorbox}

For further information on using \verb|amsmath| see \href{https://www.overleaf.com/learn/latex/Aligning_equations}{our help article}.

\begin{tcolorbox}
\begin{verbatim}
    \documentclass{article}
    \usepackage{amsmath}% For the equation* environment
    \begin{document}
    \section{First example}

    The well-known Pythagorean theorem \(x^2 + y^2 = z^2\) was 
    proved to be invalid for other exponents, meaning the next
    equation has no integer solutions for \(n>2\):

    \[ x^n + y^n = z^n \]

    \section{Second example}

    This is a simple math expression \(\sqrt{x^2+1}\) inside text. 
    And this is also the same: 
    \begin{math}
    \sqrt{x^2+1}
    \end{math}
    but by using another command.

    This is a simple math expression without numbering
    \[\sqrt{x^2+1}\] 
    separated from text.

    This is also the same:
    \begin{displaymath}
    \sqrt{x^2+1}
    \end{displaymath}

    \ldots and this:
    \begin{equation*}
    \sqrt{x^2+1}
    \end{equation*}
    \end{document}
\end{verbatim}
\end{tcolorbox}

This example produces the following output:

\begin{mdframed}
    \-\hspace{20pt}\textbf{\Large 1  First example}
    
    \-\hspace{20pt}The well-known Pythagorean theorem \(x^2 + y^2 = z^2\) was proved to be invalid for other exponents, meaning the next equation has no integer solutions for \(n>2\):

    \[ x^n + y^n = z^n \]

    \-\hspace{20pt}\textbf{\Large 2  Second example}

    \-\hspace{20pt}This is a simple math expression \(\sqrt{x^2+1}\) inside text. 
    And this is also the same: 
    \begin{math}
    \sqrt{x^2+1}
    \end{math}
    but by using another command.

    \-\hspace{20pt}This is a simple math expression without numbering
    \[\sqrt{x^2+1}\] 
    separated from text.

    \-\hspace{20pt}This is also the same:
    \begin{displaymath}
    \sqrt{x^2+1}
    \end{displaymath}

    \ldots and this:
    \begin{equation*}
    \sqrt{x^2+1}
    \end{equation*} 
\end{mdframed}

The possibilities with math in \LaTeX\ are endless so be sure to visit our help pages for advice and examples on specific topics:

\begin{itemize}
    \item \href{https://www.overleaf.com/learn/latex/Mathematical_expressions}{Mathematical expressions}
    \item \href{https://www.overleaf.com/learn/latex/Subscripts_and_superscripts}{Subscripts and superscripts}
    \item \href{https://www.overleaf.com/learn/latex/Brackets_and_Parentheses}{Brackets and Parentheses}
    \item \href{https://www.overleaf.com/learn/latex/Fractions_and_Binomials}{Fractions and Binomials}
    \item \href{https://www.overleaf.com/learn/latex/Aligning_equations_with_amsmath}{Aligning Equations}
    \item \href{https://www.overleaf.com/learn/latex/Operators}{Operators}
    \item \href{https://www.overleaf.com/learn/latex/Spacing_in_math_mode}{Spacing in math mode}
    \item \href{https://www.overleaf.com/learn/latex/Integrals%2C_sums_and_limits}{Integrals, sums and limits}
    \item \href{https://www.overleaf.com/learn/latex/Display_style_in_math_mode}{Display style in math mode}
    \item \href{https://www.overleaf.com/learn/latex/List_of_Greek_letters_and_math_symbols}{List of Greek letters and math symbols}
    \item \href{https://www.overleaf.com/learn/latex/Mathematical_fonts}{Mathematical fonts}
\end{itemize}